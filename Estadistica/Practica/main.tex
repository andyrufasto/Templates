\documentclass{article}
\usepackage[utf8]{inputenc}
\usepackage{tikz}
\usepackage{amsmath}


\documentclass{article}
\textheight = 21cm
% largo texto impreso
\textwidth = 16cm % ancho texto impreso
\topmargin = -2cm % margen superior 3-2=1cm
\oddsidemargin = -0.5cm
% margen izquierdo 4.5-2=2.5cm
% Sangría=0mm

\parindent = 0mm
\title{Práctica}

\begin{document}

\maketitle

\begin{enumerate}
    \item Dado una población de media $\mu=40$ y varianza ${\sigma}^2=1600$ se toma una muestra de $n=35$.

    \begin{enumerate}
        \item Cuales son la media y la varianza de las distribución de medias muéstrales en el muestreo?
        
        \item Demuestre que $V(\bar{x})=\frac{\sigma}{n}$ enel muestreo de medias.
        
        \item Cuales son las probabilidades de que la media muestral se encuentre entre 98 y 101
        
        \item Cuál es el valor de la varianza muestral tal que el $5\%$ de las varianzas muestrales resultan inferiores a este valor?
        
        \item Cuál es el valor de la varianza muestral, que sólo el $5\%$ de las varianzas muestrales lo superan?
        \end{enumerate}    
    \item Se obtiene una muestra aleatória de tamaño $n=25$ de una poblacion que sigue una distribución normal de media $\mu=198$ y varianza $100$
    \begin{enumerate}
        \item Cuál es la probabilidad de que la media muestral sea superior a 200?
        \item Cuál es el valor de la varianza muestral, tal que el $5\%$ de las varianzas muestrales sean inferiores a dicho valor?
        \item Cual es el valor de la varianza muestral, tal que el $5\%$ de varianzas muestrales resulten superiores a tal varianza muestral?
        \end{enumerate}
        
    
    \item Un proceso produce lotes de un producto químico cuyas concentraciones de impurezas, siguen una istribución normal de varianza $1.75$. Se elige una muestra aleatoria de $20$ lotes.\\
        Hallar la probabilidad de que la varianza muestral sea superior a $3.10$
    \item Se quiere que las bolsas de verduras congeladas tengan un peso cuya variación sea pequeña respecto del peso indicado que debe tener el cociente de la varianza muestral entre la varianza poblacional de una muestra de 20 observaciones. Dicho limite debe ser tal que la probabilidad de que el cociente lo supere sea de 0.025 por lo tanto el $97.5\%$ de los cocientes seran inferioires a dicho limite. Suponga que la población sigue una distribución normal.
    


\end{enumerate}

\end{document}
