\documentclass{article}
\usepackage[utf8]{inputenc}
\usepackage{amsmath}
\usepackage{geometry}
\usepackage{pgfplots}
\usepackage{underoverlap}
\pgfplotsset{compat=1.15}
\usepgfplotslibrary{fillbetween}

\usepackage{graphicx}
\usepackage{multirow}
 \geometry{
 a4paper,
 total={170mm,257mm},
 left=20mm,
 top=15mm,
 }
 \title{Tarea}
\date{}
\author{Andy Rufasto}

\begin{document}
\maketitle
Se toma una muestra de bebedores de cerveza y se les pregunto su preferencia entre 3 tipos de cerveza: PILSEN, CUZQUEÑA y CRISTAL.\\
\center{RESULTADOS:}\\
\begin{table}[h]
\center
\begin{tabular}{|l|l|l|}
\hline
\multirow{2}{*}{Tipo de Cerveza} & \multicolumn{2}{c|}{Género} \\ \cline{2-3} 
                                 & Masculino     & Femenino    \\ \hline \hline
PILSEN                           & 51            & 39          \\ \hline
CUZQUEÑA                         & 56            & 21          \\ \hline
CRISTAL                          & 25            & 8           \\ \hline
\end{tabular}
\end{table}

\begin{flushleft}
El interes del investigador es determinar si la preferencia por un tipo de cerveza está determinado por el género del consumidor o es independiente del consumidor. $\alpha = 0.05$\\

$H_0$ = La preferencia por el tipo de cerveza es independiente del género del bebedor\\
$H_A$ = La preferencia por el tipo de cerveza no es independiente del género del bebedor\\

\end{flushleft}

\begin{table}[h]
\centering
\begin{tabular}{|l|l|l||l|}
\hline
\multirow{2}{*}{Tipo de Cerveza} & \multicolumn{2}{c|}{Género} &       \\ \cline{2-4} 
                                 & Masculino     & Femenino    & Total \\ \hline \hline
PILSEN                           & 51            & 39          & 90    \\ \hline
CUZQUEÑA                         & 56            & 21          & 77    \\ \hline
CRISTAL                          & 25            & 8           & 33    \\ \hline \hline
\multicolumn{1}{|r|}{Total}      & 132           & 68          & 200   \\ \hline
\end{tabular}
\caption{Valores observados}
\end{table}

\begin{table}[h]
\center
\begin{tabular}{|l|l|l|}
\hline
\multirow{2}{*}{Tipo de Cerveza} & \multicolumn{2}{c|}{Género} \\ \cline{2-3} 
                                 & Masculino     & Femenino    \\ \hline \hline
PILSEN                           & $\dfrac{132 \times 90}{200} = 59.40$ & $\dfrac{68 \times 90}{200} = 30.60$       \\ [1ex] \hline
CUZQUEÑA                         & $\dfrac{132 \times 77}{200} = 50.82$ & $\dfrac{68 \times 77}{200} = 26.18$       \\ [1ex] \hline
CRISTAL                          & $\dfrac{132 \times 33}{200} = 21.78$ & $\dfrac{68 \times 33}{200} = 11.22$       \\ [1ex] \hline
\end{tabular}
\caption{Valores esperados}
\end{table}


Estadisatico de prueba: 

$ \dfrac{(f_{ij}-e_{ij})^2}{e_{ij}} \sim \chi^2_{(k,\; 1-\alpha)}$

\newpage

\begin{table}[h!]
\centering
\begin{tabular}{ll|l|rrr|r|}
\hline
\multicolumn{1}{|l|}{Tipo} & Genero & $f_{ij}$ & \multicolumn{1}{r|}{$e_{ij}$} & \multicolumn{1}{r|}{($f_{ij} - e_{ij}$ )} & $ (f_{ij} - e_{ij})^2 $ & $\dfrac{(f_{ij}-e_{ij})^2}{e_{ij}}$ \\ [1ex] \hline \hline
\multicolumn{1}{|l|}{\multirow{2}{*}{Pilsen}}   & Masculino & 51  & \multicolumn{1}{r|}{59.40} & \multicolumn{1}{r|}{-8.4}  & 70.56 & 1.38 \\ \cline{2-7}
\multicolumn{1}{|l|}{}                          & Femenino  & 39  & \multicolumn{1}{r|}{30.60} & \multicolumn{1}{r|}{8.4}   & 70.56 & 1.80 \\ \hline
\multicolumn{1}{|l|}{\multirow{2}{*}{Cuzqueña}} & Masculino & 56  & \multicolumn{1}{r|}{50.82} & \multicolumn{1}{r|}{5.18}  & 26.83 & 0.48 \\ \cline{2-7}
\multicolumn{1}{|l|}{}                          & Femenino  & 21  & \multicolumn{1}{r|}{26.18} & \multicolumn{1}{r|}{-5.18} & 26.83 & 1.28 \\ \hline
\multicolumn{1}{|l|}{\multirow{2}{*}{Cristal}}  & Masculino & 25  & \multicolumn{1}{r|}{21.78} & \multicolumn{1}{r|}{3.22}  & 10.37 & 0.41 \\ \cline{2-7}
\multicolumn{1}{|l|}{}                          & Femenino  & 8   & \multicolumn{1}{r|}{11.22} & \multicolumn{1}{r|}{-3.22} & 10.37 & 1.30 \\ \hline
                                                &           & 200 &                            &                            &       & 6.65 \\ \cline{3-3} \cline{7-7}
\end{tabular}
\caption{Calculo del Estadístico de Prueba}
\end{table}

Siendo:\\
\begin{align*}
\alpha &= 0.05\\
	  k&= (i-1)\times (j-1)\\
	   &=(3-1)\times(2-1)\\
	   &=2\\
\chi^2 _0 &= \chi^2_{(2,\;0.95)}\\
       &= 5.99
\end{align*}

$6.65 \ge 5.99$\\

Se rechaza la hipótesis nula: No existe suficiente evidencia estadística para afirmar que La preferencia por el tipo de cerveza es independiente del genero del bebedor.

	   
\end{document}

