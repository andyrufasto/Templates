\documentclass[10pt,a4paper]{report}
\usepackage[utf8]{inputenc}
\usepackage[spanish]{babel}
\usepackage{amsmath}
\usepackage{amsfonts}
\usepackage{amssymb}
\usepackage{graphicx}


\author{\Lain}

\usepackage{geometry}
 \geometry{
 a4paper,
 total={170mm,257mm},
 left=20mm,
 top=20mm,
 }

\addto\captionsspanish{%
  \renewcommand\chaptername{Problema}}

 
\begin{document}
\chapter{}
Una empresa ha terminado de instalar una linera de producción a un costo total de S./3 600 000.00 y empezó a producir a inicios del 2012,Se fija su vida económica en 5 años. La proyección de la producción es la siguiente:\\
	
	\begin{tabular}{|c|c|c|c|c|} \hline
	AÑO 1 & AÑO 2 & AÑO 3 & AÑO 4 & AÑO 5 \\ \hline
	3 500 & 3 000 & 2 500 & 2 000 & 1 000 \\ \hline
	\end{tabular}
	
	El año 2012 se espera vender toda la producció a un precio de S./ 2 000 cada uno, el costo de ventas sin considerar la depreciación se estima en un 40\%.Los gastos estimados para el 2012 son:\\
	Gastos admnistrativos: S/ 450 000.00\\
	Gastos de venta: 12\% de las ventas.\\
	\section{Linea recta y unidades físicas}

	
	\begin{table}[h]
	\centering
	\caption{Depreciación en linea recta}	
	
	\begin{tabular}{|c|c|c|c|c|c|c|} \hline
	
	Fecha   & Costo D.  & Tasa LR & V.L. Dep  & D. Anual & D. Acum   & V. Libros \\  \hline

	01/2012 & 3 600 000 & -       & -         & -        & -         & 3 600 000 \\  \hline

	12/2012 & -         & 0.2     & 3 600 000 & 720 000  & 720 000   & 2 880 000 \\  \hline

	12/2013 & -         & 0.2     & 2 880 000 & 720 000  & 1 440 000 & 2 160 000 \\  \hline

	12/2014 & -         & 0.2     & 2 160 000 & 720 000  & 2 160 000 & 1 440 000 \\  \hline

	12/2015 & -         & 0.2     & 1 440 000 & 720 000  & 2 880 000 & 720 000   \\  \hline

	12/2016 & -         & 0.2     & 720 000   & 720 000  & 3 600 000 & -         \\  \hline
	\end{tabular}
	\end{table}
	
	%Unidades físicas%
	



\begin{align*} 
Unidades        &= 3500 + 3000 + 2 500 + 2000 + 1000 \\
                &= 12000.\\
D \times Unidad &= \dfrac{3 600 000}{12000}\\
                &= 300.
\end{align*}



	\begin{table}[h]
	\centering
	\caption{Depreciación en unidades físicas}	
	
	\begin{tabular}{|c|c|c|c|c|c|c|} \hline
	
	Fecha   & Costo D.  & $D \times Unidad$ & Unidades & D. Anual  & D. Acum   & V. Libros \\ \hline

	01/2012 & 3 600 000 & -                 & -        & -         & -         & 3 600 000 \\ \hline

	12/2012 & -         & 300               & 3 500    & 1 050 000 & 1 050 000 & 2 550 000 \\ \hline

	12/2013 & -         & 300               & 3 000    & 900 000   & 1 950 000 & 1 650 000 \\ \hline

	12/2014 & -         & 300               & 2 500    & 750 000   & 2 700 000 & 900 000   \\ \hline

	12/2015 & -         & 300               & 2 000    & 600 000   & 3 300 000 & 300 000   \\ \hline

	12/2016 & -         & 300               & 1 000    & 300 000   & 3 600 000 & -         \\ \hline
	\end{tabular}
	\end{table}
	

\begin{table}[h!]
\centering
\caption{Estado de Resultados comparados año 2012}
\begin{tabular}{l|r|r} 
                          & Linea Recta & Unid. Físicas \\ \hline
 Ventas                   & 7 000 000   & 7 000 000     \\
 Gastos Administrativos   & 450 000     & 450 000       \\
 Gastos de venta          & 840 000     & 840 000       \\
 Costos de venta          & 2 800 000   & 2 800 000     \\
 Depreciación             & 720 000     & 1 050 000     \\ \hline
 Ut. Bruta                & 2 190 000   & 1 860 000     \\
 Intereses                & 150 000     & 150 000       \\ \hline
 Util. antes de Impuestos & 2 040 000   & 1 710 000
 
\end{tabular}

\end{table}

\newpage	%%%%Nueva página%%%%%%%%%%%%%%

\section{Linea recta y doble saldo decreciente}
	 
	
	
	\begin{table}[h]
	\centering
	\caption{Depreciación en linea recta}	
	
	\begin{tabular}{|c|c|c|c|c|c|c|} \hline
	
	Fecha   & Costo D.  & Tasa LR & V.L. Dep  & D. Anual & D. Acum   & V. Libros \\ \hline

	01/2012 & 3 600 000 & -       & -         & -        & -         & 3 600 000 \\ \hline

	12/2012 & -         & 0.2     & 3 600 000 & 720 000  & 720 000   & 2 880 000 \\ \hline

	12/2013 & -         & 0.2     & 2 880 000 & 720 000  & 1 440 000 & 2 160 000 \\ \hline

	12/2014 & -         & 0.2     & 2 160 000 & 720 000  & 2 160 000 & 1 440 000 \\ \hline

	12/2015 & -         & 0.2     & 1 440 000 & 720 000  & 2 880 000 & 720 000   \\ \hline

	12/2016 & -         & 0.2     & 720 000   & 720 000  & 3 600 000 & -         \\ \hline
	\end{tabular}
	\end{table}
	%%%%%%%%%%%%%%%%%%%%%%%%%%%%%%%%%%%%%%%%%%%%%
		\begin{table}[h]
	\centering
	\caption{Depreciación en Doble Saldo Decreciente}	
	
	\begin{tabular}{|c|c|c|c|c|c|c|} \hline
	
	Fecha   & Costo D.  & Tasa DSD & V.L. Dep   & D. Anual  & D. Acum   & V. Libros \\ \hline

	01/2012 & 3 600 000 & -        & -          & -         & -         & 3 600 000 \\ \hline

	12/2012 & -         & 0.4      & 3 600 000  & 1 440 000 & 1 440 000 & 2 160 000 \\ \hline

	12/2013 & -         & 0.4      & 2 1600 000 & 864 000   & 2 304 000 & 1 296 000 \\ \hline

	12/2014 & -         & 0.4      & 1 296 000  & 518 000   & 2 822 000 & 777 600   \\ \hline

	12/2015 & -         & 0.4      & 777 600    & 311 040   & 311 040   & 466 000   \\ \hline

	12/2016 & -         & -        & 466 000    & 466 000   & 3 600 000 & -         \\ \hline
	\end{tabular}
	\end{table}
	\begin{table}[h!]
	\centering
	\caption{Estado de Resultados comparados año 2012}
	\begin{tabular}{l|r|r} 
	                          & Linea Recta & Doble Saldo Decreciente \\ \hline
	 Ventas                   & 7 000 000   & 7 000 000               \\
	 Gastos Administrativos   & 450 000     & 450 000                 \\
	 Gastos de venta          & 840 000     & 840 000                 \\
	 Costos de venta          & 2 800 000   & 2 800 000               \\
	 Depreciación             & 720 000     & 1 440 000               \\ \hline
	 Ut. Bruta                & 2 190 000   & 1 470 000               \\
	 Intereses                & 150 000     & 150 000                 \\ \hline
	 Util. antes de Impuestos & 2 040 000   & 1 320 000
 	\end{tabular}
	\end{table}	 
\newpage %%%%%%%%%%%%%%%%%%%%%%%%%%%%%%%%%%%%%%%%%%%%%%%%
\chapter{}
Una empresa nos presenta información relativa a su compra de maquinarias y equipo siguientes dos años y nos solicita preparar una comparación de las depreciaciones, asimismo presentar los Estados de Resultados incluyendo las interpretaciones sobre los efectos de la depreciación según cada método utilizado.\\
Costo de maquinaria S/ 1 500 000, vida económica 5 años, las unidades de producción presupuestadas son año 1: 250 000, año 2 350 000, año 3: 500 000, año 4 : 600 000, año 5: 300 00.\\
Las ventas proyectadas son año 1:6 000 000, año 2: 5 000 000.\\

EL costo de ventas sin incluir las depreciaciones equivalente al 70\% de las ventas, los gastos operativos equivalentes  al 10\% de las ventas de cada año respectivamente.\\

Los intereses ascienden a 50 000 anualmente. No hay datos sobre amortizaciones de deuda.\\

\section{Linea recta y Doble saldo decreciente}
	\begin{table}[h]
	\centering
	\caption{Depreciación en linea recta}	
	
	\begin{tabular}{|c|c|c|c|c|c|c|} \hline
	
	Fecha    & Costo D.  & Tasa LR & V.L. Dep  & D. Anual & D. Acum & V. Libros \\ \hline

	01/$x_1$ & 1 500 000 & -       & -         & -        & -       & 1 500 000 \\ \hline

	12/$x_1$ & -         & 0.2     & 1 500 000 & 300 000  & 300 000 & 1 200 000 \\ \hline

	12/$x_2$ & -         & 0.2     & 1 500 000 & 300 000  & 600 000 & 900 000   \\ \hline
	

	\end{tabular}
	\end{table}
	
		\begin{table}[h]
	\centering
	\caption{Depreciación en Doble Saldo Decreciente}	
	
	\begin{tabular}{|c|c|c|c|c|c|c|} \hline
	
	Fecha    & Costo D.  & Tasa DSD & V.L. Dep  & D. Anual & D. Acum & V. Libros \\ \hline

	01/$x_1$ & 1 500 000 & -        & -         & -        & -       & 1 500 000 \\ \hline

	12/$x_1$ & -         & 0.4      & 1 500 000 & 600 000  & 600 000 & 900 000   \\ \hline

	12/$x_2$ & -         & 0.4      & 900 000   & 360 000  & 960 000 & 540 000   \\ \hline
	
	
	\end{tabular}
	\end{table}
	\begin{table}[h!]
	\centering
	\caption{Estado de Resultados comparados años $x_1$ y $x_2$}
	\begin{tabular}{l|r|r||r|r} 
	                          & LR $(x_1)$ & DSD($x_1$) & LR $(x_2)$ & DSD ($x_2$) \\ \hline
	 Ventas                   & 6 000 000  & 6 000 000  & 5 000 000  & 5 000 000   \\
	 Gastos Operativos        & 600 000    & 600 000    & 500 000    & 500 000     \\
	 Costos de venta          & 4 200 000  & 4 200 000  & 3 500 000  & 3 500 000   \\
	 Depreciación             & 300 000    & 600 000    & 300 000    & 360 000     \\ \hline
	 Ut. Bruta                & 900 000    & 600 000    & 700 000    & 640 000     \\
	 Intereses                & 50 000     & 50 000     & 50 000     & 50 000      \\ \hline
	 Util. antes de Impuestos & 850 000    & 550 000    & 650 000    & 590 000
 	\end{tabular}
	\end{table}
		 
\newpage%%%%%%%%%%%%%%%%%%%%%%%%%%%%%%%%%%%%%%%%%%%%%%
\section{Linea recta y Sumatoria de los dígitos de los años}	
	\begin{table}[h]
	\centering
	\caption{Depreciación en linea recta}	
	
	\begin{tabular}{|c|c|c|c|c|c|c|} \hline
	
	Fecha    & Costo D.  & Tasa LR & V.L. Dep  & D. Anual & D. Acum   & V. Libros \\ \hline

	01/$x_1$ & 1 500 000 & -       & -         & -        & -         & 1 500 000 \\ \hline

	12/$x_1$ & -         & 0.2     & 1 500 000 & 300 000  & 300 000   & 1 200 000 \\ \hline

	12/$x_2$ & -         & 0.2     & 1 500 000 & 300 000  & 600 000   & 900 000   \\ \hline

	12/$x_3$ & -         & 0.2     & 1 500 000 & 300 000  & 900 000   & 600 000   \\ \hline

	12/$x_4$ & -         & 0.2     & 1 500 000 & 300 000  & 1 200 000 & 300 000   \\ \hline

	12/$x_5$ & -         & 0.2     & 1 500 000 & 300 000  & 1 500 000 & -         \\ \hline

	\end{tabular}
	\end{table}

\begin{table}[h]
	\centering
	\caption{Depreciación en SDA}	
	
	\begin{tabular}{|c|c|c|c|c|c|c|} \hline
	
	Fecha    & Costo D.  & Tasa SDA & V.L. Dep  & D. Anual & D. Acum   & V. Libros \\ \hline

	01/$x_1$ & 1 500 000 & -        & -         & -        & -         & 1 500 000 \\ \hline

	12/$x_1$ & -         & 0.3333   & 1 500 000 & 500 000  & 500 000   & 1 000 000 \\ \hline

	12/$x_2$ & -         & 0.2667   & 1 500 000 & 400 000  & 900 000   & 600 000   \\ \hline

	12/$x_3$ & -         & 0.2000   & 1 500 000 & 300 000  & 1 200 000 & 300 000   \\ \hline

	12/$x_4$ & -         & 0.1333   & 1 500 000 & 200 000  & 1 400 000 & 100 000   \\ \hline

	12/$x_5$ & -         & 0.0667   & 1 500 000 & 100 000  & 1 500 000 & -         \\ \hline

	\end{tabular}
	\end{table}
	
	\begin{table}[h!]
	\centering
	\caption{Estado de Resultados comparados años $x_1$ y $x_2$}
	\begin{tabular}{l|r|r||r|r} 
	                          & LR $(x_1)$ & SDA($x_1$) & LR $(x_2)$ & SDA ($x_2$) \\ \hline
	 Ventas                   & 6 000 000  & 6 000 000  & 5 000 000  & 5 000 000   \\
	 Gastos Operativos        & 600 000    & 600 000    & 500 000    & 500 000     \\
	 Costos de venta          & 4 200 000  & 4 200 000  & 3 500 000  & 3 500 000   \\
	 Depreciación             & 300 000    & 500 000    & 300 000    & 400 000     \\ \hline
	 Ut. Bruta                & 900 000    & 700 000    & 700 000    & 600 000     \\
	 Intereses                & 50 000     & 50 000     & 50 000     & 50 000      \\ \hline
	 Util. antes de Impuestos & 850 000    & 650 000    & 650 000    & 550 000
 	\end{tabular}
	\end{table}	
	
\newpage %%%%%%%%%%%%%%%%%%%%%%%%%%%%%%%%%%%%%%%%%
\chapter{}
Una empresa está por comprar una maquinaria cuyo costo total asciende a S/ 4 000 000 y su vida util proyectada es 5 años.


	
\end{document}
